\documentclass[a4paper,11pt,twoside]{report}
% THIS FILE SHOULD BE COMPILED BY pdfLaTeX

% ----------------------   PREAMBLE PART ------------------------------

% ------------------------ ENCODING & LANGUAGES ----------------------

\usepackage[utf8]{inputenc}
%\usepackage[MeX]{polski} % Not needed unless You have a name with polish symbols or sth
\usepackage[T1]{fontenc}
\usepackage[english, polish]{babel}
\usepackage{float}           % Required for [H] placement specifier


\usepackage{amsmath, amsfonts, amsthm, latexsym} % MOSTLY MATHEMATICAL SYMBOLS

\usepackage[final]{pdfpages} % INPUTING TITLE PDF PAGE - GENERATE IT FIRST!
%\usepackage[backend=bibtex, style=verbose-trad2]{biblatex}


\usepackage{commath} % various commands which can make writing math expressions easier --- documentation available at: https://ctan.gust.org.pl/tex-archive/macros/latex/contrib/commath/commath.pdf

\usepackage[hidelinks]{hyperref} % for hyperlinks, for example, urls, references to equations, entries in a bibliography --- hidelinks option removes rectangles around hiperlinks


% ---------------- MARGINS, INDENTATION, LINESPREAD ------------------

\usepackage[inner=20mm, outer=20mm, bindingoffset=10mm, top=25mm, bottom=25mm]{geometry} % MARGINS


\linespread{1.5}
\allowdisplaybreaks         % ALLOWS BREAKING PAGE IN MATH MODE

\usepackage{indentfirst}    % IT MAKES THE FIRST PARAGRAPH INDENTED; NOT NEEDED
\setlength{\parindent}{5mm} % WIDTH OF AN INDENTATION


%---------------- RUNNING HEAD - CHAPTER NAMES, PAGE NUMBERS ETC. -------------------

\usepackage{fancyhdr}
\pagestyle{fancy}
\fancyhf{}
% PAGINATION: LEFT ALIGNMENT ON EVEN PAGES, RIGHT ALIGNMENT ON ODD PAGES 
\fancyfoot[LE,RO]{\thepage} 
% RIGHT HEADER: zawartość \rightmark do lewego, wewnętrznego (marginesu) 
\fancyhead[LO]{\sc \nouppercase{\rightmark}}
% lewa pagina: zawartość \leftmark do prawego, wewnętrznego (marginesu) 
\fancyhead[RE]{\sc \leftmark}

\renewcommand{\chaptermark}[1]{\markboth{\thechapter.\ #1}{}}

% HEAD RULE - IT'S A LINE WHICH SEPARATES HEADER AND FOOTER FROM CONTENT
\renewcommand{\headrulewidth}{0 pt} % 0 MEANS NO RULE, 0.5 MEANS FINE RULE, THE BIGGER VALUE THE THICKER RULE


\fancypagestyle{plain}{
  \fancyhf{}
  \fancyfoot[LE,RO]{\thepage}
  
  \renewcommand{\headrulewidth}{0pt}
  \renewcommand{\footrulewidth}{0.0pt}
}



% --------------------------- CHAPTER HEADERS ---------------------

\usepackage{titlesec}
\titleformat{\chapter}
  {\normalfont\Large \bfseries}
  {\thechapter.}{1ex}{\Large}

\titleformat{\section}
  {\normalfont\large\bfseries}
  {\thesection.}{1ex}{}
\titlespacing{\section}{0pt}{30pt}{20pt} 

    
\titleformat{\subsection}
  {\normalfont \bfseries}
  {\thesubsection.}{1ex}{}


% ----------------------- TABLE OF CONTENTS SETUP ---------------------------

\def\cleardoublepage{\clearpage\if@twoside
\ifodd\c@page\else\hbox{}\thispagestyle{empty}\newpage
\if@twocolumn\hbox{}\newpage\fi\fi\fi}


% THIS MAKES DOTS IN TOC FOR CHAPTERS
\usepackage{etoolbox}
\makeatletter
\patchcmd{\l@chapter}
  {\hfil}
  {\leaders\hbox{\normalfont$\m@th\mkern \@dotsep mu\hbox{.}\mkern \@dotsep mu$}\hfill}
  {}{}
\makeatother

\usepackage{titletoc}
\makeatletter
\titlecontents{chapter}% <section-type>
  [0pt]% <left>
  {}% <above-code>
  {\bfseries \thecontentslabel.\quad}% <numbered-entry-format>
  {\bfseries}% <numberless-entry-format>
  {\bfseries\leaders\hbox{\normalfont$\m@th\mkern \@dotsep mu\hbox{.}\mkern \@dotsep mu$}\hfill\contentspage}% <filler-page-format>

\titlecontents{section}
  [1em]
  {}
  {\thecontentslabel.\quad}
  {}
  {\leaders\hbox{\normalfont$\m@th\mkern \@dotsep mu\hbox{.}\mkern \@dotsep mu$}\hfill\contentspage}

\titlecontents{subsection}
  [2em]
  {}
  {\thecontentslabel.\quad}
  {}
  {\leaders\hbox{\normalfont$\m@th\mkern \@dotsep mu\hbox{.}\mkern \@dotsep mu$}\hfill\contentspage}
\makeatother



% ---------------------- TABLES AD FIGURES NUMBERING ----------------------

\renewcommand*{\thetable}{\arabic{chapter}.\arabic{table}}
\renewcommand*{\thefigure}{\arabic{chapter}.\arabic{figure}}


% ------------- DEFINING ENVIRONMENTS FOR THEOREMS, DEFINITIONS ETC. ---------------

\makeatletter
\newtheoremstyle{definition}
{3ex}%                           % Space above
{3ex}%                           % Space below
{\upshape}%                      % Body font
{}%                              % Indent amount
{\bfseries}%                     % Theorem head font
{.}%                             % Punctuation after theorem head
{.5em}%                          % Space after theorem head, ' ', or \newline
{\thmname{#1}\thmnumber{ #2}\thmnote{ (#3)}}
\makeatother

\theoremstyle{definition}
\newtheorem{theorem}{Theorem}[chapter]
\newtheorem{lemma}[theorem]{Lemma}
\newtheorem{example}[theorem]{Example}
\newtheorem{proposition}[theorem]{Proposition}
\newtheorem{corollary}[theorem]{Corollary}
\newtheorem{definition}[theorem]{Definition}
\newtheorem{remark}[theorem]{Remark}

% --------------------- END OF PREAMBLE PART (MOSTLY) --------------------------





% -------------------------- USER SETTINGS ---------------------------

\newcommand{\tytul}{Projekt i wdrożenie kolaboratywnej tablicy z wykorzystaniem podejścia „local-first"}
\renewcommand{\title}{Design and Implementation of a Collaborative Board Using the
Local-First Approach}
\newcommand{\type}{Engineer} % Master OR Engineer
\newcommand{\supervisor}{dr inż. Paweł Kotowski} % TITLE AND NAME OF THE SUPERVISOR



\begin{document}
\sloppy
\selectlanguage{english}

\includepdf[pages=-]{titlepage-en} % THIS INPUTS THE TITLE PAGE

\null\thispagestyle{empty}\newpage

% ------------------ PAGE WITH SIGNATURES --------------------------------

%\thispagestyle{empty}\newpage
%\null
%
%\vfill
%
%\begin{center}
%\begin{tabular}[t]{ccc}
%............................................. & \hspace*{100pt} & .............................................\\
%supervisor's signature & \hspace*{100pt} & author's signature
%\end{tabular}
%\end{center}
%


% ---------------------------- ABSTRACTS -----------------------------

{  \fontsize{12}{14} \selectfont
\begin{abstract}

\begin{center}
\title
\end{center}
The purpose of this thesis is to design and implement a collaborative board using the local-first approach. 
The solution uses distribuited conflict free replicated data types (CRDTs) and real-time communication via WebRTC protocol to bring
effortless collaboration with other users. \\
% TODO: one sentance summarizing each chapter - FIRST WRITE CHAPTERS
% TODO: general view of the work done
Final result of the effort done is an desktop application that enables drawing and erasing on the whiteboard with mutiple users
simultaneously. This paper covers description of solution, but also thouches broader topic of local-first application design architectures.
\\
\\
\noindent \textbf{Key words: rust language, local-first, CRDT, WebRTC, distributed system, star topology, multithread real time communication} 
\end{abstract}
}

\null\thispagestyle{empty}\newpage


{\selectlanguage{polish} \fontsize{12}{14}\selectfont
\begin{abstract}

\begin{center}
\tytul
\end{center}
Celem pracy inynierskiej jest zaprojektowanie i implementacja kolaboratywnej tablicy
 z wykorzystaniem podejścia "local-first".
 Wypracowane rozwiązanie uzywa rozproszonego bezkonfliktowego typy danych (CRDTs) oraz komunikacji w 
 czasie rzeczywistym przez protokół WebrRTC by umozliwic bezproblemową kolaborację z innymi uzytkownikami.
 \\
 Ostatecznym wynikiem jest aplikacja desktop-owa umozliwiająca rysowanie oraz zmazywanie poprzednich 
 pociągnięć na białej tablicy z wieloma uzytkownikami jednocześnie. Praca pokrywa opis rozwiązania 
 ale równiez porusza bardziej obszerny temat architektur typu local-first w aplikacjach kolaboratywnych.


\noindent \textbf{Słowa kluczowe:} język rust, local-first, CRDT, WebRTC, system rozproszony, topologia gwiazdy, wielowątkowa komunikacja w czasie rzeczywistym
\end{abstract}
}


%% --------------------------- DECLARATIONS ------------------------------------
%
%%
%%	IT IS NECESSARY OT ATTACH FILLED-OUT AUTORSHIP DEECLRATION. SCAN (IN PDF FORMAT) NEEDS TO BE PLACED IN scans FOLDER AND IT SHOULD BE CALLED, FOR EXAMPLE, DECLARATION_OF_AUTORSHIP.PDF. IF THE FILENAME OR FILEPATH IS DIFFERENT, THE FILEPATH IN THE NEXT COMMAND HAS TO BE ADJUSTED ACCORDINGLY.
%%
%%	command attacging the declarations of autorship
%%
%\includepdf[pages=-]{scans/declaration-of-autorship}
%\null\thispagestyle{empty}\newpage
%
%% optional declaration
%%
%%	command attaching the declaataration on granting a license
%%
%\includepdf[pages=-]{scans/declaration-on-granting-a-license}
%%
%%	.tex corresponding to the above PDF files are present in the 3. declarations folder 
%
\null\thispagestyle{empty}\newpage
% ------------------- TABLE OF CONTENTS ---------------------
% \selectlanguage{english} - for English
\pagenumbering{gobble}
\tableofcontents
\thispagestyle{empty}
\newpage % IF YOU HAVE EVEN QUANTITY OD PAGES OF TOC, THEN REMOVE IT OR ADD \null\newpage FOR DOUBLE BLANK PAGE BEFORE INTRODUCTION


% -------------------- THE BODY OF THE THESIS --------------------------------

\null\thispagestyle{empty}\newpage
\pagestyle{fancy}
\pagenumbering{arabic}
\setcounter{page}{11}


\chapter{Introduction}
\markboth{}{Introduction}
\addcontentsline{toc}{chapter}{Introduction}

% DONT ADD DESCRIPTION OF SOLUTION

\section{Analysis and description of collaborative local-first applications}

% Describe the problem and analyze its context here.
Collaborative applications have not always been so popular as they are these days. You might not 
even know that some of applications that we use every day are collaborative. In everyday life of a 
software developer almost all of modern IDEs have collaborative work options, almost all project management 
software are also collaborative. These are examples close to software engineers, but these applications do 
not limit themselves to software development. Google Docs, Microsoft 365, Figma, Miro and many more are examples of
collaborative applications that are used by millions of people every day. These applications are not only used for work, 
but also for education and entertainment. 

It hasn't always been like this. The steep rise of collaborative applications began as the remote work became 
more popular but also it owes its rise of popularity to universality of fast global network and increase of computing power
of everyday devices. 

Current collaborative apps couldn't exist 20 years ago. They haven't been simply invented yet. Typical 
application of early days of online applications had a simple architecture. It was a client-server application. 
The server was responsible for storing data and processing it, while the client was responsible for presenting the 
data to the user and sending user input to the server. 
This architecture had its advantages and disadvantages. It was simple and easy to implement, but it had a single point of
failure.
If the server went down, the entire application went down. It also had scalability issues, as the server had to handle all the requests from the clients.

% Place image exactly here
\begin{figure}[H]
	\centering
	\includegraphics[width=0.3\textwidth]{./images/simple-architecture.png}
	\caption{Simple client-server architecture of early internet era applications}
	\label{fig:simple-architecture}
\end{figure}

These kind of application belong to ``cloud-first'' family. There was no
need for ``local-first'' architectures yet and this term has not been popularized. Before even the 
discussion on topic of  ``local-first architectures'' the ``cloud-first'' approach had its own more than one 
decade 
of fast development. From simple client server architectures the system rose to more difficult forms of 
large size client-side application with lots of layers and microservices. Picture below shows architectures 
that emerged 
on the market with rise of popularity of mobile and web apps from 2010.

% Place image exactly here
\begin{figure}[H]
	\centering
	\includegraphics[width=0.3\textwidth]{./images/more-complicated-architecture.png}
	\caption{More complicated architecture of modern mobile/web applications}
	\label{fig:more-complicated-architecture}
\end{figure}

How far are these architectures from beeing good fit for collaboarative appliactions? Lets have a look on
 typical modules that take part in standrd user interface interaction. Picure below shows typical modules of 
 morden web/mobile/desktop cloud-application.

% Place image exactly here
\begin{figure}[H]
	\centering
	\includegraphics[width=1.0\textwidth]{./images/complicated-modules.png}
	\caption{Typical modules of modern cloud-based applications}
	\label{fig:complicated-modules}
\end{figure}


As we can see the level of complication if very high. Simple user intercation with a system 
triggers multiple functions from multiple modules from multiulpe services. This level of 
sophistication simply cannot be used for collaborative applications, because orchestration 
of synchoronization of state shown in simultaneus users would be extremely difficult. 

For one, in would require to many asynchronous time-consuming remote calls to different services. 
Total synchronization nightmare. For some time tough there has been efforts to implement collaboration 
this way. This approach was known as OT - operational transformation (still used in Google Docs). Although Google Docs is a 
good example of usage of cloud-first approach, in this thesis I decided to focus on local-first architecutres
as they have proved to be more broadly used in the industry and it offers way more possibilities of different 
kind of applications. 

There are also my other reasons to avoid cloud-first architectures when designing collaborative applications. 
Lets take a look at simple sequance of user interaction with the system. Below we can see general asynchronous 
interaction of a user with normal cloud-basen application. 
\begin{figure}[H]
	\centering
	\includegraphics[width=0.7\textwidth]{./images/typical-sequance.png}
	\caption{Typical sequance of interaction with a user in a modern cloud-based applications}
	\label{fig:typical-sequence}
\end{figure}


As we can see on on this diagram, the system relies heavily on connectivity with the database. The big question that comes 
comes to mind is - what happens to the system if there is no internet ? Answer is simple. It would not work. Therefore,
one could even say, that whole application is just a asynchronous graphical database wrapper, and it wouldn't 
be far from truth. This brings us the quastion - What is the definition of cloud-first and local first applications?

In simplest words the definition of cloud-approach is "If the interaction of the user with the system haven't been 
saved to database, it never happened". On other hand local-first is "What's important the most is local persistance - 
cloud brings only connectivity".

Therefore let's have a look on how would typical sequance of user interaction with 
local-first system look. The diagram below shows typical sequance in such system.

\begin{figure}[H]
	\centering
	\includegraphics[width=0.7\textwidth]{./images/local-first-sequanec.png}
	\caption{Typical sequance of interaction with a user in a local-first applications}
	\label{fig:local-first-sequence}
\end{figure}


The distinction is clear. In local-first approach the user interaction with the system is not dependent 
on connectivity with the database. Connectivity brings only messaging with other peers. Whole 
synchronization of replicas of data between users happens on local machine. This is the reaseon 
why we can call this system distributed. We can also notice that this approach requires much less
modules across different serivces. Later, in the thesis, we will clearly how this concept allows
us to bring true real time collaboration to the users.


\section{Local-first vs Cloud-first}

Lets have a look on the main differences between local-first and cloud-first approaches. 
The table below shows the main differences between these two approaches when it comes to designing 
collaborative application.

\begin{figure}[H]
	\centering
	\includegraphics[width=0.7\textwidth]{./images/cloud-vs-local.png}
	\caption{Cloud vs Local comparison}
	\label{fig:cloud-vs-local}
\end{figure}

Clearly, the obvious choice for apprach when designing collaborative application would be Local-first. 
Let's now try to  distinguish, what are boundaries of such system - what use cases are not applicable with 
local-first apprach. The table below shows use cases that are not good fit for local-first approach.

\begin{table}[H]
	\centering
	\caption{Use cases: Suitable and Unsuitable for Local-First Approach}
	\begin{tabular}{|p{0.45\textwidth}|p{0.45\textwidth}|}
		\hline
		\textbf{Suitable for} & \textbf{Unsuitable for} \\
		\hline
		Real-time collaborative editing (documents, whiteboards) & Applications requiring strong centralized control \\
		Offline-first mobile/desktop apps & Systems with strict regulatory or audit requirements \\
		Peer-to-peer file sharing & Applications with massive global data aggregation \\
		Personal productivity tools with sync & Centralized transactional systems (e.g., banking) \\
		Small/medium team collaboration & Apps needing immediate global consistency \\
		\hline
	\end{tabular}
\end{table}



\section{Why would one implement own collaborative system ?}

Creating collaborative whiterboard is just an intriduction to the general topic of distrubuted conflict 
free real time data storage with editable operations. Designign whiteboard is easy exaple for system that
can be used to many use cases (distribited conflict free data) - it serves as a  great introduction to
 this topic. 

 There are multiple use cases/possible usage of distrubuted systems. For example drone swarms - distributed replicas 
 of formation among drones broadcasted between each other. Peer-to-peer file sharing -
  distributed replicas of files among users. Real time collaborative editing - distributed
   replicas of documents among users. 

\section{Examples from industry}

Known examples of local-first applications are: Figma, Miro, Notion, Obsidian, Roam Research, Coda, Google
 Docs (although it is not pure local-first application).
 These applications are used by millions of people every day. They are used for work, education and 
 entertainment.


\section{Known architectures for collaborative applications}


\begin{figure}[H]
	\centering
	\includegraphics[width=0.8\textwidth]{./images/p2p-vs-start.png}
	\caption{Comparison of Client-Server and Peer-to-Peer (P2P) Architectures}
	\label{fig:client-server-vs-p2p}
\end{figure}

Connectivity among multiple users is always solved eigher by peer-to-peer or more "centered" relay 
architecure for example start-topology. When designig a collaborative system each solution has its
benefits and setbacks. In a table belowe there is simple analysis of pros and cons of choosing one or the ohter
when designing local-first collaborative whiteboard.


\begin{table}[H]
	\centering
	\caption{Comparison of Peer-to-Peer and Star Topology Architectures}
	\begin{tabular}{|p{0.45\textwidth}|p{0.45\textwidth}|}
		\hline
		\textbf{Peer-to-Peer (P2P)} & \textbf{Star Topology} \\
		\hline
		No need to create a server for communication. Communication is direct - good. Clients manage their connections themselves - might be problematic.
		& Central node relays communication. It requires support of server. Keeping servers running is expansive.\\
		\hline
		Communication might be faster because it is direct. Users might be close to each other. &
		Clients sometimes might unnesesarry communicate via very distant server to both of them, which would cause delays. \\
		\hline
		High resilience to single node failure & Central node is a single point of failure \\
		\hline
		Complex synchronization and conflict resolution - corrdination of connections has to be implemented on client side. 
		& Easier management and coordination - server keeps all informations on connectivity. Clients connect only to one server. \\
		\hline
		Scalability can be challenging - many users might have very low maximum of connections it could sustain. & 
		Scalability depends on central node capacity, but server can easily be scaled hirizontlay or verticaly\\
		\hline
		Suitable for decentralized systems with limited numer of peers at the same time (for example 10) 
		& Suitable for medium (up to 50) teams with central coordination - for example video conferance rooms \\
		\hline
	\end{tabular}
\end{table}

As we can see - each solution suits different applications.


\section{OT vs CRDTs}

One very important topic in the matter of conflict resolution and collaboration is data structures algorightms 
that can be used. Local / remote inserts into data structures need to be \textbf{commutative}, \textbf{associative} and \textbf{idempotent}.
These three properties are required to ensure that all replicas of data will be eventually consistent - assuming that all 
operations (inserts/deletes) propagate properly. There are two main
approaches to achieve this - OT (operational transformation) and CRDTs (conflict-free replicated data types).

Historically well known solution is OT - operational transformations. The basic idea
behind this approach is to transform operations in such a way that they can be applied in any order and still produce the same result.
This approach was used in Google Docs and it is still used in some applications. However, it has some drawbacks. It is complex to implement and
 it can lead to conflicts that are difficult to resolve.

CRDTs - conflict-free replicated data types - are a more recent approach to achieve eventual consistency in distributed systems.
The basic idea behind CRDTs is to design data structures in such a way that they can be updated independently and still converge
 to the same state. The algorithms are implemented directly into data structures. This approach is simpler to implement and it can
  handle conflicts more gracefully. It is used in many modern applications, including Figma, Miro, Notion, Obsidian, Roam Research, 
  Coda and many more.

% Provide motivating examples.

\section{Vision of the System}

The envisioned system is a desktop application designed for real-time collaborative drawing on a whiteboard.
 It leverages the Rust programming language for performance and reliability, and utilizes the egui library 
 to provide a modern, responsive user interface. Communication between users is achieved through WebRTC,
  with the Livekit open-source server facilitating room-based connections. Each room acts as a channel 
  or document, allowing users to join by simply entering a room code or name, without the need for user 
  accounts, authentication, or authorization.

The application employs CRDTs (Conflict-Free Replicated Data Types) from the Automerge library to ensure 
seamless synchronization of whiteboard state across all participants. This approach enables users to draw
 and erase simultaneously, with changes instantly reflected for everyone in the room. The cursors of other
   users are visible, enhancing the collaborative experience. The whiteboard will be represented as a canvas 
   with raster graphics of strokes from users as vector of points with certain width and color.

For real-time, multithreaded communication, the system uses the Tokio library, ensuring efficient handling
 of concurrent operations. Notably, the application does not persist data by default; once the app is 
 closed, all data is lost unless explicitly exported to a file by the user. The focus is on desktop
  environments, with no mobile version planned.

\begin{figure}[H]
	\centering
	\includegraphics[width=0.8\textwidth]{./images/architecture.png}
	\caption{Architecture of the envisioned system}
	\label{fig:architecture}
\end{figure}


\section{Requirements Specification}


\subsection{Functional Requirements}

Below you can see tables with user stories for each actor in the system: user, application, and server.
\begin{table}[H]
	\centering
	\caption{Functional Requirements -- User Stories (Actor: User)}
	\begin{tabular}{|c|p{0.12\textwidth}|p{0.38\textwidth}|p{0.3\textwidth}|}
		\hline
		\textbf{ID} & \textbf{As a...} & \textbf{I want to...} & \textbf{In order to...} \\
		\hline
		US-01 & User & launch the desktop app and see blank whiteboard &  start drawing immediately without any setup \\
		\hline
		US-02 & User & draw strokes on whiteboard & \\
		\hline
		US-03 & User & & \\
		\hline
		US-04 & User & & \\
		\hline
		US-05 & User & & \\
		\hline
	\end{tabular}
	\label{tab:user-stories-user}
\end{table}

\begin{table}[H]
	\centering
	\caption{Functional Requirements -- User Stories (Actor: Application)}
	\begin{tabular}{|c|p{0.12\textwidth}|p{0.38\textwidth}|p{0.3\textwidth}|}
		\hline
		\textbf{ID} & \textbf{As a...} & \textbf{I want to...} & \textbf{In order to...} \\
		\hline
		AP-01 & Application & & \\
		\hline
		AP-02 & Application & & \\
		\hline
		AP-03 & Application & & \\
		\hline
		AP-04 & Application & & \\
		\hline
		AP-05 & Application & & \\
		\hline
	\end{tabular}
	\label{tab:user-stories-application}
\end{table}

\begin{table}[H]
	\centering
	\caption{Functional Requirements -- User Stories (Actor: Server)}
	\begin{tabular}{|c|p{0.12\textwidth}|p{0.38\textwidth}|p{0.3\textwidth}|}
		\hline
		\textbf{ID} & \textbf{As a...} & \textbf{I want to...} & \textbf{In order to...} \\
		\hline
		SV-01 & Server & & \\
		\hline
		SV-02 & Server & & \\
		\hline
		SV-03 & Server & & \\
		\hline
		SV-04 & Server & & \\
		\hline
		SV-05 & Server & & \\
		\hline
	\end{tabular}
	\label{tab:user-stories-server}
\end{table}

% List and describe the functional requirements of the system.
\subsection{Non-Functional Requirements}

% List and describe the non-functional requirements of the system.

\section{Business Cases}

% Describe business cases and the added value.

\chapter{Example chapter}

This \TeX~file is to be compiled with pdfLaTeX (it's just quick build in TeXMaker).


\section{Example section}

\begin{definition}[Definition]
	%
	A \emph{definition} is a statement of the meaning of a term (a word, phrase, or other set of symbols).
	%
\end{definition}

\subsection{Example subsection}

It's the deepest deph of sectioning allowed by rector.

\begin{definition}[Equation]
	%
	In mathematics, an \emph{equation} is a statement of an equality containing one or more variables.
	%
\end{definition}
%
\begin{example}
This is an example of an equation:
\begin{equation}
	%
	2+2=4.
	%
\end{equation}
%
Equation without a number:
\begin{equation*}
	%
	2+2=4,
	%
\end{equation*}
or:
\[
2+2=4.
\]

It is worthwhile to peruse other mathematical environments like  \emph{multline}, \emph{align} and their versions with a star (, i.e. without numeration). The description of their use can be found at  \url{https://texdoc.org/serve/amsldoc.pdf/0} starting from the end of the third page.

Equation \eqref{equation} is false. References (and some other things) work properly after compliling \TeX ~ file twice.

\begin{equation}\label{equation}
	%
	\int \limits_{0}^{1} x \; dx = \frac{3}{2}.
	%
\end{equation}

\end{example}

Theorem \ref{Pitagoras} is a very interensting result.

\begin{theorem}[Pythagoras' Theorem]\label{Pitagoras}
	%
	Let $c$ represent the length of the hypotenuse and $a$ and $b$ the lengths of the triangle's other two sides. Then:
	%
\begin{equation*}
	%
	a^2 + b^2 = c^2.
	%
\end{equation*}
%
\end{theorem}

\begin{proof}
	%
	The proof has been presented in \cite{Ktos} and \cite{Innyktos}. We can write then \cite{Ktos, Innyktos}.
	%
\end{proof}

\begin{corollary}
	%
	The use of the term \emph{corollary}, rather than \emph{proposition} or \emph{theorem}, is intrinsically subjective.
	%
\end{corollary}




\begin{remark}
	%
	You can find a rather comprehensive list of available symbols at \url{https://www3.nd.edu/~nmark/UsefulFacts/LaTeX_symbols.pdf}.
	
	If you want to find a symbol by its shape, you can use the following site: \url{https://detexify.kirelabs.org/classify.html}. 
	%
\end{remark}

\begin{lemma}[Someone's Lemma]
	%
	Ten lemat jest nie na temat.
	%
\end{lemma}
\begin{proof} Dowód przez indukcję.
\end{proof}


Lorem ipsum dolor sit amet, consetetur sadipscing elitr, sed diam nonumyeirmod tempor invidunt ut labore et dolore magna aliquyam erat, sed diamvoluptua. At vero eos et accusam et justo duo dolores et ea rebum. Stet clita kasd gubergren, no sea takimata sanctus est Lorem ipsum dolor sit amet.Lorem ipsum dolor sit amet, consetetur sadipscing elitr, sed diam nonumyeirmod tempor invidunt ut labore et dolore magna aliquyam erat, sed diamvoluptua. At vero eos et accusam et justo duo dolores et ea rebum. Stet clita kasd gubergren, no sea takimata sanctus est Lorem ipsum dolor sit amet.



\section{Floats -- tables and figures}

\begin{table}% label has to be after caption, otherwise numeration is wrong
\caption[Short caption]{Additional options}
\label{opcje}
\centering
\begin{tabular}{|c|p{0.8\textwidth}|}
\hline
symbol & effect \\ \hline
\texttt{h} & Place the float here, i.e., approximately at the same point it occurs in the source text (however, not exactly at the spot) \\
\texttt{t} & Position at the top of the page \\
\texttt{b} & Position at the bottom of the page \\
\texttt{p} & Put on a special page for floats only \\
\texttt{!} & Override internal parameters LaTeX uses for determining "good" float positions \\ 
\texttt{H} & Places the float at precisely the location in the \LaTeX ~ code. Requires the float package,[1] i.e., \texttt{\textbackslash usepackage\{float\}}. This is somewhat equivalent to !ht.\\ \hline
\end{tabular}
\end{table}

Place labels after captions or you get the wrong labelling.

In Table \ref{opcje} there are additional options for \texttt{table} and \texttt{figure} environments.

\begin{figure}[h!]

\begin{center}
    \setlength{\unitlength}{1mm}

    \begin{picture}(40, 30)
        \put(20,1){\line(0,1){20}} % line

        % bottom
        \put(20,1){\circle*{2}}
        \put(25,1){0}

        % top
        \put(20,21){\circle*{2}}
        \put(25,21){1}
    \end{picture}

\end{center}
\caption{Example figure -- it has been drawn by \LaTeX ~default tools}
\end{figure}


Lorem ipsum dolor sit amet, consetetur sadipscing elit, sed diam nonumyeirmod tempor invidunt ut labore et dolore magna aliquyam erat, sed diamvoluptua. At vero eos et accusam et justo duo dolores et ea rebum. Stet clita kasd gubergren, no sea takimata sanctus est Lorem ipsum dolor sit amet.Lorem ipsum dolor sit amet, consetetur sadipscing elitr, sed diam nonumyeirmod tempor invidunt ut labore et dolore magna aliquyam erat, sed diamvoluptua. At vero eos et accusam et justo duo dolores et ea rebum. Stet clita kasd gubergren, no sea takimata sanctus est Lorem ipsum dolor sit amet.




\chapter{The next chapter}

Lorem ipsum dolor sit amet, consetetur sadipscing elit, sed diam nonumyeirmod tempor invidunt ut labore et dolore magna aliquyam erat, sed diamvoluptua. At vero eos et accusam et justo duo dolores et ea rebum. Stet clita kasd gubergren, no sea takimata sanctus est Lorem ipsum dolor sit amet.Lorem ipsum dolor sit amet, consetetur sadipscing elitr, sed diam nonumyeirmod tempor invidunt ut labore et dolore magna aliquyam erat, sed diamvoluptua. At vero eos et accusam et justo duo dolores et ea rebum. Stet clita kasd gubergren, no sea takimata sanctus est Lorem ipsum dolor sit amet.


\section{Matrices}

Simple matrix:
\begin{equation*}
	\begin{matrix}
	a & b & c & d \\
	d & e & f & g \\
	1 & 1 & 1 & 1
	\end{matrix}
\end{equation*}
%
Matrix with parentheses:
%
\begin{equation*}
	A = 
	\begin{pmatrix}
	a & b & c & d \\
	d & e & f & g \\
	1 & 1 & 1 & 1
	\end{pmatrix}
\end{equation*}
%
Matrix with brackets:
%
\begin{equation*}
	\begin{bmatrix}
	a & b & c & d \\
	d & e & f & g \\
	1 & 1 & 1 & 1
	\end{bmatrix}
\end{equation*}
%
You can also use more general environment:
%
\begin{equation*}
	\renewcommand{\arraystretch}{0.8}
	\begin{array}{ccc}
	1 & 0 & 0 \\
	0 & 1 & 0 \\
	0 & 0 & 1 \\
	\end{array}
\end{equation*}
%
Matrix with braces:
%
\begin{equation*}
	\left\{
	\renewcommand{\arraystretch}{0.8}
	\begin{array}{ccc}
	1 & 0 & 0 \\
	0 & 1 & 0 \\
	0 & 0 & 1 \\
	\end{array}\right\}
\end{equation*}

\begin{definition}
	%
	Let $A\neq \emptyset$, $n \in \mathbb{N}$. Every function $f\colon A^n \to A$ is called an \emph{$n$-ary operation} or \emph{działaniem} określonym na $A$.
	$0$-ary operations are constant functions.
	%
\end{definition}


\begin{definition}[Algebra]
	%
	The ordered pair $(A,F)$, where $A\neq \emptyset$ is a set and $F$ is a family of operations defined on $A$, shall be called an \emph{algebra} (or \emph{$F$-algebra}). The set $A$ is called \emph{the set of elements}, \emph{support} or \emph{universe} of an algebra $(A,F)$ and $F$ is called \emph{the set of elementary operations}.
	%
\end{definition}

\begin{proposition}
	%
	I state that, having passed to the limit, the only thing left me me is to camp at said limit or return, or, maybe, search for a pass or an exit to other areas.
	%
\end{proposition}





% ------------------------------- BIBLIOGRAPHY ---------------------------
% LEXICOGRAPHICAL ORDER BY AUTHORS' LAST NAMES
% FOR AMBITIOUS ONES - USE BIBTEX


\begin{thebibliography}{20} % IF YOU HAVE MORE REFERENCES, WRITE THE BIGGER NUMBER

\bibitem[1]{Ktos} A. Author, \emph{Title of a book}, Publisher, year, page--page.
\bibitem[2]{Innyktos} J. Bobkowski, S. Dobkowski, Title of an article, \emph{Magazine X, No. 7}, year, PAGE--PAGE.
\bibitem[3]{B} C. Brink, Power structures, \emph{Algebra Universalis 30(2)}, 1993, 177--216.
\bibitem[4]{H} F. Burris, H. P. Sankappanavar, \emph{A Course of Universal Algebra}, Springer-Verlag, New York, 1981.
\end{thebibliography}
\pagenumbering{gobble}
\thispagestyle{empty}



% ----------------------- LIST OF SYMBOLS AND ABBREVIATIONS ------------------
\chapter*{List of symbols and abbreviations}

\begin{tabular}{cl}
nzw. & nadzwyczajny \\
* & star operator \\
$\widetilde{}$ & tilde 
\end{tabular}
\\
If you don't need it, delete it.
\thispagestyle{empty}


% ----------------------------  LIST OF FIGURES --------------------------------
\listoffigures
\thispagestyle{empty}
If you don't need it, delete it.


% -----------------------------  LIST OF TABLES --------------------------------
\renewcommand{\listtablename}{Spis tabel}
\listoftables
\thispagestyle{empty}
If you don't need it, delete it.

% -----------------------------  LIST OF APPENDICES ---------------------------
\chapter*{List of appendices}
\begin{enumerate}
\item Appendix 1
\item Appendix 2
\item In case of no appendices, delete this part.
\end{enumerate}
\thispagestyle{empty}


\end{document}
